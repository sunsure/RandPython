\documentclass[a4paper,12pt]{article}
%%%%%%%%%%%%%%%%%%%%%%%%%%%%%%%%%%%%%%%%%%%%%%%%%%%%%%%%%%%%%%%%%%%%%%%%%%%%%%%%%%%%%%%%%%%%%%%%%%%%%%%%%%%%%%%%%%%%%%%%%%%%%%%%%%%%%%%%%%%%%%%%%%%%%%%%%%%%%%%%%%%%%%%%%%%%%%%%%%%%%%%%%%%%%%%%%%%%%%%%%%%%%%%%%%%%%%%%%%%%%%%%%%%%%%%%%%%%%%%%%%%%%%%%%%%%
\usepackage{eurosym}
\usepackage{vmargin}
\usepackage{amsmath}
\usepackage{graphics}
\usepackage{epsfig}
\usepackage{framed}
\usepackage{subfigure}
\usepackage{fancyhdr}

\setcounter{MaxMatrixCols}{10}
%TCIDATA{OutputFilter=LATEX.DLL}
%TCIDATA{Version=5.00.0.2570}
%TCIDATA{<META NAME="SaveForMode"CONTENT="1">}
%TCIDATA{LastRevised=Wednesday, February 23, 201113:24:34}
%TCIDATA{<META NAME="GraphicsSave" CONTENT="32">}
%TCIDATA{Language=American English}

\pagestyle{fancy}
\setmarginsrb{20mm}{0mm}{20mm}{25mm}{12mm}{11mm}{0mm}{11mm}
\lhead{Neural Networks} \rhead{Kevin O'Brien} \chead{ULCIS} %\input{tcilatex}

%http://www.electronics.dit.ie/staff/ysemenova/Opto2/CO_IntroLab.pdf
\begin{document}

\tableofcontents

\section{About RPy2}

rpy2 is a redesign and rewrite of rpy. It is providing a low-level interface to R, a proposed high-level interface, including wrappers to graphical libraries, as well as R-like structures and functions.

\subsection{Documentation}

rpy2 has a growing documentation, with documentation pages for each released or developped version.

Contact: The rpy mailing-list, questions/answers boards such as stackoverflow should be used for most of the questions.Use the contacts page otherwise.
\newpage
%-----------------------------------------------------------------------------------------%

\section{Quick Tutorial on RPy Package for R/Python Interface}

%Professor Norm Matloff
%Dept. of Computer Science
%University of California at Davis
%Davis, CA 95616

Contents of this site:

Installing RPy
Introduction to using RPy
What is RPY?

%-----------------------------------------------------------------------------------------%
\subsection{What is RPy?}
RPy is a simple, easy-to-use interface to R from Python. It enables one to enjoy the elegance of Python programming while having access to the rich graphical and statistical capabilities of R.

In its simplest form, shown here, one includes in one's Python code a statement

from rpy2.robjects import r
This launches an execution of R, with communication to the original Python program. The Python class instance r includes various functions for remote execution of R commands, including those involved with data produced by the Python program.

IMPORTANT NOTE: The material here concerns RPy2, not the original RPy.

%-----------------------------------------------------------------------------------------%
\subsection{Installing RPY}


Dowload RPy from the RPY home page. Unpack it, and in the top directory created by the package, open a shell/command window and run
\begin{verbatim}
python setup.py install
\end{verbatim}

If you are on a multiuser system and do not have root privileges, you can specify a nondefault root directory. For example, on the UC Davis Computer Science Department's instructional machines, I typed
%--------------------------------------------------------------------------------------------%
R RHOME
setenv RHOME /usr/lib/R
python setup.py install --root /home/matloff/Pub/rpy2
The first command ran R with a request to report where R was installed on the system, which turned out to be /usr/lib/R. The second command set the corresponding shell environment variable (C shell in my case). The third command specified a nondefault installation directory.

\section{Introduction to using RPY:}

First, make sure the RPy module is in your Python path. In the above context, I typed
\begin{verbatim}
setenv PYTHONPATH /home/matloff/Pub/rpy2/usr/lib/python2.5/site-packages/
\end{verbatim}
Now, let's generate vectors x and y in R, do a scatter plot, fit a least-squares line, etc.:

\begin{verbatim}
>>> from rpy2.robjects import r
>>> r('x <- rnorm(100)')  # generate x at R
>>> r('y <- x + rnorm(100,sd=0.5)')  # generate y at R
>>> r('plot(x,y)')  # have R plot them
>>> r('lmout <- lm(y~x)')  # run the regression
>>> r('print(lmout)')  # print from R
>>> loclmout = r('lmout') # download lmout from R to Python
>>> print loclmout  # print locally
>>> print loclmout.r['coefficients']  # print one component
\end{verbatim}
%----------------------------------------------------------------------------------------%
Now let's apply some R operations to some Python variables:

\begin{verbatim}
>>> u = range(10)  # set up another scatter plot, this one local
>>> e = 5*[0.25,-0.25]
>>> v = u[:]
>>> for i in range(10): v[i] += e[i]
>>> r.plot(u,v)
>>> r.assign('remoteu',u)  # ship local u to R
>>> r.assign('remotev',v)  # ship local v to R
>>> r('plot(remoteu,remotev)')  # plot there
\end{verbatim}
There are many more functions. See the RPy documentation for details.

%---------------------------------------------------------------------------------------------------------%
\newpage
\section{A Demonstration of RPy: Tim Churches}

RPy, written by Walter Moreira and maintained by Gregory Warnes, is a Python extension module for using the R programming environment for data analysis and graphics from within Python.

RPy is available from the RPy project web page. As Walter notes, RPy was inspired by RSPython by Duncan Temple Lang. RSPython allows R to be called from Python and vice-versa (i.e. Python can be embedded in R), as well as providing more general facilities for exploiting the object-oriented aspects of both Python and R. However, at least for me, RPy is a lot easier to use. It is my sincere hope that RPy and RSPython can be merged in a display of el Norte/el Sur co-operation, so we can have the best of both.

The following example provides a small taste of both the power of the R environment and the ease with which RPy allows this power to be used from within Python.
\subsubsection{Old Faithful Example}
The data are eruption times for the Old Faithful geyser which, along with Yogi Bear, is located in the Yellowstone National Park in Wyoming, USA. The data file faithful.dat was exported from the faithful example dataset which comes as part of R. The R code in this example was borrowed directly from Section 8.2 of "An Introduction to R, Version 1.4.1" by W.N. Venables, D.M. Smith and the R Development Core Team. Minimal changes to the orginal R code were required to make it work from within Python, thanks to RPy.

The following Python code (faithful.py):
\begin{verbatim}
from rpy import *

faithful_data = {"eruption_duration":[],
                 "waiting_time":[]}
		
f = open('faithful.dat','r')

for row in f.readlines()[1:]: # skip the column header line
    splitrow = row[:-1].split(" ")
    faithful_data["eruption_duration"].append(float(splitrow[0]))
    faithful_data["waiting_time"].append(int(splitrow[1]))

f.close()

ed = faithful_data["eruption_duration"]
edsummary = r.summary(ed)
print "Summary of Old Faithful eruption duration data"
for k in edsummary.keys():
    print k + ": %.3f" % edsummary[k]
print
print "Stem-and-leaf plot of Old Faithful eruption duration data"
print r.stem(ed)

r.png('faithful_histogram.png',width=733,height=550)
r.hist(ed,r.seq(1.6, 5.2, 0.2), prob=1,col="lightgreen",
       main="Old Faithful eruptions",xlab="Eruption duration (seconds)")
r.lines(r.density(ed,bw=0.1),col="orange")
r.rug(ed)
r.dev_off()

long_ed = filter(lambda x: x > 3, ed)
r.png('faithful_ecdf.png',width=733,height=550)
r.library('stepfun')
r.plot(r.ecdf(long_ed), do_points=0, verticals=1, col="blue",
       main=paste("Empirical cumulative distribution function",
                  " of Old Faithful eruptions longer than 3 seconds")
x = r.seq(3,5.4,0.01)
r.lines(r.seq(3,5.4,0.01),r.pnorm(r.seq(3,5.4,0.01),mean=r.mean(long_ed),
        sd=r.sqrt(r.var(long_ed))), lty=3, lwd=2, col="red")
r.dev_off()

\begin{verbatim}
r.png('faithful_qq.png',width=733,height=550)
r.par(pty="s")
r.qqnorm(long_ed,col="blue")
r.qqline(long_ed,col="red")
r.dev_off()

r.library('ctest')
print
print("Shapiro-Wilks normality test of Old Faithful eruptions" +\
      " longer than 3 seconds")
sw = r.shapiro_test(long_ed)
print "W = %.4f" % sw['statistic']['W']
print "p-value = %.5f" % sw['p.value']

print
print("One-sample Kolmogorov-Smirnov test of Old Faithful eruptions" +\
      " longer than 3 seconds"
ks = r.ks_test(long_ed,"pnorm", mean=r.mean(long_ed),
               sd=r.sqrt(r.var(long_ed)))
print "D = %.4f" % ks['statistic']['D']
print "p-value = %.4f" % ks['p.value']
print "Alternative hypothesis: %s" % ks['alternative']
print
\end{verbatim}
produces the following output:

Summary of Old Faithful eruption duration data
Mean: 3.488
Median: 4.000
3rd Qu.: 4.454
1st Qu.: 2.163
Min.: 1.600
Max.: 5.100
\end{verbatim}
Stem-and-leaf plot of Old Faithful eruption duration data
\begin{verbatim}
  The decimal point is 1 digit(s) to the left of the |

  16 | 070355555588
  18 | 000022233333335577777777888822335777888
  20 | 00002223378800035778
  22 | 0002335578023578
  24 | 00228
  26 | 23
  28 | 080
  30 | 7
  32 | 2337
  34 | 250077
  36 | 0000823577
  38 | 2333335582225577
  40 | 0000003357788888002233555577778
  42 | 03335555778800233333555577778
  44 | 02222335557780000000023333357778888
  46 | 0000233357700000023578
  48 | 00000022335800333
  50 | 0370

None

Shapiro-Wilks normality test of Old Faithful eruptions longer than 3 seconds
W = 0.9793
p-value = 0.01052

One-sample Kolmogorov-Smirnov test of Old Faithful eruptions longer than 3 seconds
D = 0.0661
p-value = 0.4284
Alternative hypothesis: two.sided
\end{verbatim}
%and these graphs:
\newline
%------------------------------------------------------------------------------------------------------------%


\end{document}
