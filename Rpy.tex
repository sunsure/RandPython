\documentclass[a4paper,12pt]{article}
%%%%%%%%%%%%%%%%%%%%%%%%%%%%%%%%%%%%%%%%%%%%%%%%%%%%%%%%%%%%%%%%%%%%%%%%%%%%%%%%%%%%%%%%%%%%%%%%%%%%%%%%%%%%%%%%%%%%%%%%%%%%%%%%%%%%%%%%%%%%%%%%%%%%%%%%%%%%%%%%%%%%%%%%%%%%%%%%%%%%%%%%%%%%%%%%%%%%%%%%%%%%%%%%%%%%%%%%%%%%%%%%%%%%%%%%%%%%%%%%%%%%%%%%%%%%
\usepackage{eurosym}
\usepackage{vmargin}
\usepackage{amsmath}
\usepackage{graphics}
\usepackage{epsfig}
\usepackage{framed}
\usepackage{subfigure}
\usepackage{fancyhdr}

\setcounter{MaxMatrixCols}{10}
%TCIDATA{OutputFilter=LATEX.DLL}
%TCIDATA{Version=5.00.0.2570}
%TCIDATA{<META NAME="SaveForMode"CONTENT="1">}
%TCIDATA{LastRevised=Wednesday, February 23, 201113:24:34}
%TCIDATA{<META NAME="GraphicsSave" CONTENT="32">}
%TCIDATA{Language=American English}

\pagestyle{fancy}
\setmarginsrb{20mm}{0mm}{20mm}{25mm}{12mm}{11mm}{0mm}{11mm}
\lhead{Neural Networks} \rhead{Kevin O'Brien} \chead{ULCIS} %\input{tcilatex}

%http://www.electronics.dit.ie/staff/ysemenova/Opto2/CO_IntroLab.pdf
\begin{document}

\tableofcontents

\section{About RPy2}

rpy2 is a redesign and rewrite of rpy. It is providing a low-level interface to R, a proposed high-level interface, including wrappers to graphical libraries, as well as R-like structures and functions.

\subsection{Documentation}

rpy2 has a growing documentation, with documentation pages for each released or developped version.

Contact: The rpy mailing-list, questions/answers boards such as stackoverflow should be used for most of the questions.Use the contacts page otherwise.
\newpage
%-----------------------------------------------------------------------------------------%

\section{Quick Tutorial on RPy Package for R/Python Interface}

%Professor Norm Matloff
%Dept. of Computer Science
%University of California at Davis
%Davis, CA 95616

Contents of this site:

Installing RPy
Introduction to using RPy
What is RPY?

%-----------------------------------------------------------------------------------------%
\subsection{What is RPy?}
RPy is a simple, easy-to-use interface to R from Python. It enables one to enjoy the elegance of Python programming while having access to the rich graphical and statistical capabilities of R.

In its simplest form, shown here, one includes in one's Python code a statement

from rpy2.robjects import r
This launches an execution of R, with communication to the original Python program. The Python class instance r includes various functions for remote execution of R commands, including those involved with data produced by the Python program.

IMPORTANT NOTE: The material here concerns RPy2, not the original RPy.

%-----------------------------------------------------------------------------------------%
\subsection{Installing RPY}


Dowload RPy from the RPY home page. Unpack it, and in the top directory created by the package, open a shell/command window and run
\begin{verbatim}
python setup.py install
\end{verbatim}

If you are on a multiuser system and do not have root privileges, you can specify a nondefault root directory. For example, on the UC Davis Computer Science Department's instructional machines, I typed
%--------------------------------------------------------------------------------------------%
R RHOME
setenv RHOME /usr/lib/R
python setup.py install --root /home/matloff/Pub/rpy2
The first command ran R with a request to report where R was installed on the system, which turned out to be /usr/lib/R. The second command set the corresponding shell environment variable (C shell in my case). The third command specified a nondefault installation directory.

\section{Introduction to using RPY:}

First, make sure the RPy module is in your Python path. In the above context, I typed
\begin{verbatim}
setenv PYTHONPATH /home/matloff/Pub/rpy2/usr/lib/python2.5/site-packages/
\end{verbatim}
Now, let's generate vectors x and y in R, do a scatter plot, fit a least-squares line, etc.:

\begin{verbatim}
>>> from rpy2.robjects import r
>>> r('x <- rnorm(100)')  # generate x at R
>>> r('y <- x + rnorm(100,sd=0.5)')  # generate y at R
>>> r('plot(x,y)')  # have R plot them
>>> r('lmout <- lm(y~x)')  # run the regression
>>> r('print(lmout)')  # print from R
>>> loclmout = r('lmout') # download lmout from R to Python
>>> print loclmout  # print locally
>>> print loclmout.r['coefficients']  # print one component
\end{verbatim}
%----------------------------------------------------------------------------------------%
Now let's apply some R operations to some Python variables:

\begin{verbatim}
>>> u = range(10)  # set up another scatter plot, this one local
>>> e = 5*[0.25,-0.25]
>>> v = u[:]
>>> for i in range(10): v[i] += e[i]
>>> r.plot(u,v)
>>> r.assign('remoteu',u)  # ship local u to R
>>> r.assign('remotev',v)  # ship local v to R
>>> r('plot(remoteu,remotev)')  # plot there
\end{verbatim}
There are many more functions. See the RPy documentation for details.
\end{document}
